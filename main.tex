\documentclass{beamer}
\usepackage[sorting=none]{biblatex}
\usepackage[ruled]{algorithm2e}
\usepackage{graphicx}
\usepackage{subcaption}


\usetheme{Boadilla} 
\usecolortheme{beaver}    % RED ASH WHITE
% \useinnertheme{rectangles}
\useinnertheme{circles}
\usefonttheme{structurebold}

\addbibresource{sample.bib} %Imports bibliography file

\title[The Multicut Problem]{The Multicut Problem}
\author[1805032]{
    Mahdee Mushfique Kamal, 
    \texttt{1805032} 
}
\institute[CSE,BUET]{
    Department of Computer Science and Engineering, \\
    Bangladesh University of Engineering and Technology,
    Dhaka, 1000. 
}
\date{\today}

\begin{document}


\begin{frame}
    \maketitle
\end{frame}

\setbeamercovered{transparent}

% slide 1 - introduction
\begin{frame}{Introduction}
\begin{itemize}
    \item Given a graph \(G = (V, E)\) with nonnegative edge costs \(c_e \geq 0\), the Multicut Problem considers a set of source-sink pairs \((s_1, t_1), ..., (s_k, t_k)\). \newline

    \item The goal is to find a minimum-cost set of edges \(F\) such that removing them disconnects all pairs in the graph. \newline

    \item That is, for every $ i, 1 \leq i \leq k $, there is no path connecting \(s_i\) and \(t_i\) in (V, E - F).
\end{itemize}
\end{frame}

\begin{frame}{Introduction}
\begin{itemize}
    \item The problem is slightly different from k-multiway cut problem. \newline

    \item For k=1, it is a s-t flow problem which can be solved in polynomial time. \newline

    \item Minimum multicut problem is NP-hard and MAX-SNP-hard even for k = 3. \cite{dahlhaus1994complexity}

    
\end{itemize}

\end{frame}

\begin{frame}{Essential Terminologies}
\begin{itemize}
    \item \textit{Linear programming relaxation} is a technique used in optimization to simplify a discrete, combinatorial, or integer programming problem by relaxing some of its constraints or integrality requirements. \newline

    \item We solve the linear program to obtain a fractional solution, then round it to an integer solution via some process. \newline

    \pause

    \item A \textit{separation oracle} takes as input a supposedly feasible solution $x$ to the linear program, and either verifies that $x$ is indeed a feasible solution to the linear program or, if it is infeasible, produces a constraint that is violated by $x$. \footnote{More details on section 4.3} 

\end{itemize}  
\end{frame}

 



\begin{frame}{Integer Programming Formulation}

Given \(G = (V, E)\) with nonnegative edges  We need to minimize, 

\[
\sum_{e \in E} c_e x_e
\]

Subject to:

\[
\sum_{e \in P} x_e \geq 1, \quad \forall P \in P_i, \quad 1 \leq i \leq k \quad \text{ and } \quad x_e \in \{0, 1\},  \forall e \in E
\] \newline

\begin{itemize}
    \item Here, \(x_e\) is a variable indicating whether edge \(e\) is selected or not.

    \item And, $P_i$ is the set of all paths $P$ joining $s_i$ and $t_i$. Which means, we can't remove the whole path (at least one edge of the path is selected) 

    \end{itemize}
\end{frame}


\begin{frame}{Linear Programming Relaxation}
    \begin{itemize}
        \item Replace the constraints $ x_e \in \{0, 1\} $ with $ x_e \geq 0 $ \newline

        \item We can solve the linear program in polynomial time using the \textbf{ellipsoid method} \footnote{More details on section 4.3}  \newline

        \item For the separation oracle, for each i, $ 1 \geq i \geq k $, the shortest path between $ s_i$ and $t_i$ is computed. We return "violation constraint" if shortest path $ P $ is less than 1. 

        % \item Since we have $ \sum_{e \in P} x_e < 1 $ for $ P \in P_i$ 
    \end{itemize}
    
\end{frame}


\begin{frame}{LP Rounding Algorithm}

The solution is built by taking balls around the vertex $s_i$ for each $i$. The notations for the algorithm: 

\begin{itemize}
  \item $ x_e $ denotes the length of the edge $e$. 
  \item $d_x(u,v)$ denotes the length of the shortest path from $u$ to $v$ using $x_e$ as edge lengths. $ d_x(s_i, t_i) \geq 1 $ for all $i$. 
  \pause
  \item $r$ denotes the radius of the ball. 
  \item $ B_x(s_i, r) = { v \in V : d_x(s_i, v) \leq r} $ denotes the ball of radius of r around $s_i$. 
\end{itemize}
\end{frame}


\begin{frame}{LP Rounding Algorithm}

\begin{itemize}
    \item Each edge $e \in E$ is considered as a pipe with length $x_e$ and cross-sectional area $c_e$.

    \item The product $c_e x_e$ gives the volume of edge e.

    \item The LP produces the minimum-volume system of pipes such that $d_x(s_i , t_i ) \geq 1$ for all i. 

    \begin{figure}
        \centering
        \begin{subfigure}[b]{0.45\textwidth}
            \includegraphics[width=\textwidth]{fig_8.4.png}
            \caption{Pipe system}
            \label{fig:subfig1}
        \end{subfigure}
        \hfill
        \begin{subfigure}[b]{0.45\textwidth}
            \includegraphics[width=\textwidth]{fig_8.5.png}
            \caption{Ball inside a pipe system}
            \label{fig:subfig2}
        \end{subfigure}
        \label{fig:main}
    \end{figure}


\end{itemize}
    
\end{frame}


\begin{frame}{LP Rounding Algorithm}

\begin{itemize}
  \item $ V^* = \sum_{e \in E } c_e x_e$ denotes the total volume of the pipes.

  \item $ V_x(s_i, r) $ denotes the volume of pipes within distance $r$ of $s_i$. 



\begin{align*}
    V_x(s_i, r) = & \frac{V^*}{k} + \sum_{e = (u,v) : u,v \in B_x(s_i, r)}{c_ex_e} \\
    & + \sum_{e = (u,v) : u \in B_x(s_i, r), v \notin B_x(s_i, r)}{c_e - d_x(s_i, u)} \notag \newline
\end{align*}

\pause
\item $\delta(S)$  denotes the set of edges that has exactly one endpoint in the set of vertices S.

\end{itemize}
\end{frame}

\begin{frame}{LP Rounding Algorithm}

\begin{itemize}
    \item $ c (\delta(B_x(s_i , r ))) $ denotes the cost of the edges in $\delta(B_x(s_i , r ))$. \newline 
    That is $ c (\delta(B_x(s_i , r ))) = \sum_{e \in \delta(B_x(s_i , r ))} c_e$ \newline

    \pause

    \item The paper claims that, it is always possible to find some radius $r < \frac{1}{2}$ such that the cost $ c(\delta(B_x(s_i , r ))) $ of the cut induced by the ball of radius r around $s_i$ is not much more than the volume of the ball. \newline

    \item Finding such ball is called as \textit{region growing}. 
\end{itemize}
    
\end{frame}


\begin{frame}{LP Rounding Algorithm}

\begin{lemma}[8.7]
    Given a feasible solution to the linear program x, for any $s_i$ one can find
in polynomial time a radius  $r < \frac{1}{2}$ such that

\begin{equation*}
    c(\delta(B_x(s_i , r ))) \leq (2ln(k+1))V_x(s_i, r)
\end{equation*}
\end{lemma}
\end{frame}


\begin{frame}{LP Rounding Algorithm}
  \begin{algorithm}[H]
    \SetAlgoLined

    Let $x$ be an optimal solution to the LP \\
    $F \leftarrow \emptyset$ \\
    
    \For{$i \leftarrow 1$ \KwTo $k$}{
      \If{$s_i$ and $t_i$ are connected in $(V, E - F)$}{
        Choose radius $r < \frac{1}{2}$ around $s_i$ as in Lemma 8.7 \\
        $F \leftarrow F \cup \delta(B_x(s_i, r))$ \\
        Remove $B_x(s_i, r)$ and incident edges from the graph
      }
    }
    
    \Return{$F$}
    \caption{Algorithm for the multicut problem}
  \end{algorithm}
\end{frame}


\begin{frame}{LP Rounding Algorithm}

\begin{lemma}[8.8]
    The algorithm produces a feasible solution for the multicut problem. 
\end{lemma}

\vspace{10pt}

\textit{Proof by Contradiction:} 
The only possible way in which the solution might not be feasible is if we have some $s_j - t_j$ pair in the ball $B_x(s_i , r)$ when the vertices in the ball get removed
from the graph.\newline

However, if $s_j , t_j \in B_x(s_i , r) $ for $r < \frac{1}{2}$, then $d_x(s_i , s_j ) < \frac{1}{2}$ and
$d_x(s_i , t_j ) < \frac{1}{2}$ \newline

This means, $d_x (s_j , t_j ) < 1$ which contradicts the feasibility of the LP solution x. 
    
\end{frame}

\begin{frame}{LP Rounding Algorithm}

\begin{theorem}
    The algorithm is a $4ln(k+1)$ -approximation algorithm for multicut problem. 
\end{theorem}

\vspace{10pt}

Let $B_i$ be the set of vertices in the ball and $F_i$ be the edges in $\delta(B_i)$ when the pair $s_i - t_i$ are selected for separation .

Let $V_i$ be the total volume of edges removed when the vertices of $B_i$ and its incident edges are removed from the graph.  $V_i \geq V_x(s_i, r) - \frac{v^*}{k} $.

From lemma 8.7, we also get, $c(F_i) \leq (2 \ln(k + 1)) V_x(s_i, r) \leq (2 \ln(k + 1))( V_i + \frac{v^*}{k} )$.

\end{frame}

\begin{frame}{LP Rounding Algorithm}

\begin{theorem}
    The algorithm is a $4ln(k+1)$ -approximation algorithm for multicut problem. 
\end{theorem}

\vspace{10pt}

Each volume of edge belongs to at most one $V_i$, once the edge is removed it can't be part of another volume in later itteration. 

\begin{align*}
    \sum_{e \in F} c_e = & \sum_{i=1}^k \sum_{e \in F_1} c_e \leq (2 \ln(k + 1)) \sum_{i=1}^k( V_i + \frac{v^*}{k} ) \\
    & \leq (4 \ln(k + 1)) V^* \leq (4 \ln(k + 1)) OPT
\end{align*}

\end{frame}

\begin{frame}{References}
    \printbibliography[heading=bibintoc,title={Whole bibliography}]
\end{frame}


\end{document}
